\documentclass{article}

\title{DTL - Assignment:1 - Sections; Paragraphs;
Table of Content.}
\author{Prathamesh Agawane - (142203001)}
\date {29-12-2022}

\begin{document}

\maketitle
\newpage


\tableofcontents 


\newpage
\section{\underline{Microsoft Tools:}}
\underline{\emph{Sections and Paragraphs}}
	
\subsection{MS Word:}
\paragraph{MS Word:}
MS Word usually refers to Microsoft Word. Generally, it is used as an umbrella term for all word processors that directly show you what you will get as an end result (as opposed to first having to process the file). This approach is more intuitive, but it makes editing large projects very complicated. Everyone knows Word. However, “knowing” Word mostly refers to the ease of use, as it is a “what you see is what you get” (WYSIWYG) text editor. But if I asked how, using Word, to refer to another document’s text block and add that as a citation in a footnote, most people would have to look on the Internet to find out how that could be done. While most of the functionality is available through icons, you still need to know where to look when something is not a standard command like those used in formatting, making lists, or choosing fonts.

\subsection{MS Excel:}
\paragraph{MS Excel:}
Excel is a spreadsheet program from Microsoft and a component of its Office product group for business applications. Microsoft Excel enables users to format, organize and calculate data in a spreadsheet.By organizing data using software like Excel, data analysts and other users can make information easier to view as data is added or changed. Excel contains a large number of boxes called cells that are ordered in rows and columns. Data is placed in these cells. Excel is a part of the Microsoft Office and Office 365 suites and is compatible with other applications in the Office suite. The spreadsheet software is available for Windows, macOS, Android and iOS platforms.

\subsection{MS Powerpoint:}
\paragraph{MS Powerpoint:}
PowerPoint is the leading multimedia presentation software. PowerPoint is a high-powered software tool used for presenting information in a dynamic slide show format. Text, charts, graphs, sound effects and video are just some of the elements PowerPoint can incorporate into your presentations with ease. 



\newpage
\section{Latex:}
\paragraph{Latex:}

 LaTeX is a typesetting system that works more like a compiler than a word processor. While initially complicated, LaTeX allows better management of larger projects like theses or books by splitting the document into sections: style, references, and text. In LaTeX (pronounced LAH-tekh or LAY-tekh), you instead create a text document which is then translated into an actual formatted document (your book). Formatting is done through commands you enter as text into the document. To write a LaTeX document, you never have to touch your mouse, as you can enter everything by keystrokes alone.


\subsection{Most Frequently Used Latex Commands}

\paragraph{LaTeX Commands}

A LaTeX command begins with the command name, which consists of a \ followed by either (a) a string of letters or (b) a single non-letter. Arguments contained in square brackets [] are optional while arguments contained in braces {} are required.

Note: LaTeX is case sensitive. Enter all commands in lower case unless explicitly directed to do otherwise.

\end{document}
